\documentclass[conference]{IEEEtran} 
% or: \documentclass[journal]{IEEEtran}

% -------------------------
% Packages
% -------------------------
\usepackage{amsmath,amssymb,amsthm}
\usepackage{graphicx}
\usepackage{cite}
\usepackage{mathtools}
\usepackage{bm}

% -------------------------
% Theorem environments
% -------------------------
\newtheorem{theorem}{Theorem}
\newtheorem{lemma}{Lemma}
\newtheorem{proposition}{Proposition}
\newtheorem{corollary}{Corollary}

% -------------------------
% Title
% -------------------------
\title{Descriptive and Searchable Paper Title}

\author{
\IEEEauthorblockN{Author Name}
\IEEEauthorblockA{Affiliation\\
Email}
}

\begin{document}
\maketitle

% -------------------------
% Abstract
% -------------------------
\begin{abstract}
Briefly describe:
(1) the problem,
(2) the main result,
(3) a key conclusion,
(4) an application or example (if any).
No citations, no equations.
\end{abstract}

% -------------------------
% Introduction
% -------------------------
\section{Introduction}

% Paragraph 1: problem description
Textual description of the problem and context.

% Paragraph 2: literature positioning
Discussion of closely related work and limitations.

% Paragraph 3: contributions
The main contribution of this paper is ...

% Paragraph 4: paper organization
The remainder of this paper is organized as follows...

\paragraph*{Notation}
Optional. Define symbols used throughout the paper.

% -------------------------
% Related Work (optional)
% -------------------------
\section{Related Work}
Grouped and balanced discussion of relevant literature.

% -------------------------
% Problem Statement
% -------------------------
\section{Problem Statement}

% 1. System description
Describe the system and state variables.

% 2. Dynamics / actuation
\begin{equation}
\dot{x} = f(x,u,w)
\end{equation}

% 3. Sensing / information structure
Describe outputs, measurements, information constraints.

% 4. Objective
Clearly state the control/optimization objective.

% -------------------------
% Main Results
% -------------------------
\section{Main Results}

\begin{theorem}[Main Result Title]
Consider the system described in Section~III.
Under Assumptions~1--3, the following property holds...
\end{theorem}

Short explanation of the theorem and its implications.
Avoid proofs here.

% -------------------------
% Discussion / Corollaries
% -------------------------
\section{Discussion}

\begin{corollary}
A special case or consequence of the main theorem.
\end{corollary}

Explain why this is interesting or useful.

% -------------------------
% Proofs
% -------------------------
\section{Proofs}

\begin{proof}[Proof of Theorem 1]
Explain the structure of the proof before technical steps.
Detailed derivations.
\end{proof}

% -------------------------
% Applications / Examples
% -------------------------
\section{Applications and Examples}

Describe simulations, benchmarks, or real-world scenarios.

\begin{figure}[t]
\centering
\includegraphics[width=0.9\columnwidth]{example.pdf}
\caption{Illustrative example of the proposed method.}
\end{figure}

% -------------------------
% Conclusions
% -------------------------
\section{Conclusions and Future Work}

Summarize main results and outline open problems.

% -------------------------
% Appendix
% -------------------------
\appendices
\section{Auxiliary Proofs}
Lengthy or technical material.

% -------------------------
% References
% -------------------------
\bibliographystyle{IEEEtran}
\bibliography{references}

\end{document}
