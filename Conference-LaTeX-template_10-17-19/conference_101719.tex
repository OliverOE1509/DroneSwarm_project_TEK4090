\documentclass[conference]{IEEEtran}
\IEEEoverridecommandlockouts
% The preceding line is only needed to identify funding in the first footnote. If that is unneeded, please comment it out.
\usepackage{cite}
\usepackage{amsmath,amssymb,amsfonts}
\usepackage{algorithmic}
\usepackage{graphicx}
\usepackage{textcomp}
\usepackage{graphicx}
\usepackage{xcolor}
\def\BibTeX{{\rm B\kern-.05em{\sc i\kern-.025em b}\kern-.08em
    T\kern-.1667em\lower.7ex\hbox{E}\kern-.125emX}}
\begin{document}

\title{Capture the flag with drone swarms\\
{\footnotesize \textsuperscript{*}Note: Sub-titles are not captured in Xplore and
should not be used}
\thanks{Identify applicable funding agency here. If none, delete this.}
}

\author{\IEEEauthorblockN{1\textsuperscript{st} Daniel Grønhaug}
\IEEEauthorblockA{\textit{Department of Mathematics} \\
\textit{University of Oslo}\\
Oslo, Norway \\
danieltfg@math.uio.no}
\and
\IEEEauthorblockN{2\textsuperscript{nd} Oliver Ekeberg}
\IEEEauthorblockA{\textit{Department of Mathematics} \\
\textit{University of Oslo}\\
Oslo, Norway \\
oliveroe@math.uio.no}

}

\maketitle

\begin{abstract}
This paper presents a decentralized control strategy for a multi-agent unmanned aerial vehicle (UAV) swarm tasked wiht a capture-the-flag (CTF) objective, formulated as a point reaching problem with collision avoidance. An artificial potential field (APF)-based controller is empliyed to generate velocity references for each agent through sums of attractive and repulsive vectors. A leader-follower structure is introduced, where the agent equipped with status is directly attracted to the drone, while the followers objective is to remain passive. This paper explores how changing the targets gps location, gives a more dynamic bypassing of leader status in between the drone swarm. This paper provides a Github repository with guidlines to launch the simulation from a containerized environment using Docker, and results from said simulations will be discussed.
\end{abstract}

\begin{IEEEkeywords}
Artificial potential fields, multi-agent systems, drone swarms, capture-the-flag, ROS2, Webots
\end{IEEEkeywords}

\section{Introduction}
Cooperative control of multi-agent systems is a central topic in swarm robotics, with applications ranging from surveillance, exploration to transport and search tasks. A benchmark for evaluating such strategies, is capture-the-flag problems, in which agents are required to communicate their motion towards a shared objective, while avoiding collisions and adapting to dynamic changes in the goal. When implemented using UAVs, this task poses a challenge to decentralized decision making, scalability and effectiveness of the swarm. 

Artificial Potential Fields (APF) provides a compitationally cheap and efficient approach to decentralized motion planning, by generating control actions off of sums of attraction from targets, and repulsion forces from objects to avoid colliding with. This is a 

In this paper, we formulate the multi-agent capture-the-flag (CTF) problem as a decentralized control task and propose and APF based control strategy for generating trajectories to reach said flag. A dynamic leader follower structure is introduced, where a single agent is assigned a status as leader and is directly attracted to the flag, while the remaining agents act as followers and maintain collision avoidance. This approach is implemented and evaluated in a physics-based simulation environment using Webots and ROS 2

\section{Problem Formulation}

This section formalizes the multi-agent CTF task considered, and specified sysem assumptions, architecture of nodes and topics and control objectives. The formulation is intentionally kept at a high-level in order to emphasize coordination problem, rather than low-level vehicle dynamics.

\subsection{Scenario description}

We consider a swarm of autonomus UAVs operating in a 3D virtual world, where simulated physics are the closest approach to first principles in the real world. The collective task is to capture tha flag, which is modelled as reaching a point in this space. Upon capture, the flag will respawn in a new location, represented as a vector with uniformly distributed components. This respawn mechanism, requires the swarm to adapt its behaviour. The agents are required to decide on consensus in order to reach the target efficiently, while avoiding collisions with eachother, or potentially foreign objects. There is no central trajectory planner that is assumed, and each agent computes its control action independently based locally available information.


\subsection{Agent Representation and Control Level}

Each UAV is treated as an independent agent, controlled at velocity-command level. Low level attitude stabilization and motor thrust actuators are assumed to be handled by onboard flight controller and are not explicitly modeled in this paper. As a result, the controller considered in this paper generates velocity references, rather than thrust commands. This allows the paper to focus on multi-agent coordination and interaction, rather than on detailed vehicle dynamics, which would require a longer paper. Roll and pitch dynamics are neglected, but are possible to tweak as a contribution to the github project. Only planar motion and yaw regulation are considered.

\begin{center}
    \includegraphics[width = 1 \linewidth]{assets/UAV_yaw_pitch_roll.png}
\end{center}

\subsection{Information Structure and Decentralization}

Since control architecture is decentralized, each agent has access to its own state, and to the relative states of other agents in the swarm, in addition to the position of the flag. This may be seen as a oversimplification, but it can be substituted


\section{Artificial Potential Field Controller}


\subsection{Equations}


\section{System Architecture and Implementation}

\section{Experimental Results}

\section{Discussion}

\section{Conclusion and Future Work}



\section*{References}

Please number citations consecutively within brackets \cite{b1}. The 
sentence punctuation follows the bracket \cite{b2}. Refer simply to the reference 
number, as in \cite{b3}---do not use ``Ref. \cite{b3}'' or ``reference \cite{b3}'' except at 
the beginning of a sentence: ``Reference \cite{b3} was the first $\ldots$''

Number footnotes separately in superscripts. Place the actual footnote at 
the bottom of the column in which it was cited. Do not put footnotes in the 
abstract or reference list. Use letters for table footnotes.



\begin{thebibliography}{00}
\bibitem{b1} G. Eason, B. Noble, and I. N. Sneddon, ``On certain integrals of Lipschitz-Hankel type involving products of Bessel functions,'' Phil. Trans. Roy. Soc. London, vol. A247, pp. 529--551, April 1955.
\bibitem{b2} J. Clerk Maxwell, A Treatise on Electricity and Magnetism, 3rd ed., vol. 2. Oxford: Clarendon, 1892, pp.68--73.
\bibitem{b3} I. S. Jacobs and C. P. Bean, ``Fine particles, thin films and exchange anisotropy,'' in Magnetism, vol. III, G. T. Rado and H. Suhl, Eds. New York: Academic, 1963, pp. 271--350.
\bibitem{b4} K. Elissa, ``Title of paper if known,'' unpublished.
\bibitem{b5} R. Nicole, ``Title of paper with only first word capitalized,'' J. Name Stand. Abbrev., in press.
\bibitem{b6} Y. Yorozu, M. Hirano, K. Oka, and Y. Tagawa, ``Electron spectroscopy studies on magneto-optical media and plastic substrate interface,'' IEEE Transl. J. Magn. Japan, vol. 2, pp. 740--741, August 1987 [Digests 9th Annual Conf. Magnetics Japan, p. 301, 1982].
\bibitem{b7} M. Young, The Technical Writer's Handbook. Mill Valley, CA: University Science, 1989.
\end{thebibliography}

\end{document}
